\section{Metodología}
En primera instancia, al iniciar el algoritmo se crea un set con tamaño n de coordenadas completamente aleatorias, este set después es ordenado de manera ascendente para poder dividir de manera correcta el set completo en 2 subsets.
Cabe recalcar que para que haga la división de 2 subsets tienen que presentarse más de 3 coordenadas por que si esto no es así, con el algoritmo de fuerza bruta ya será posible encontrar el par más cercano. No obstante, si se crean los 2 subsets, cada uno se le aplicará el algoritmo de fuerza bruta para poder así encontrar el par más cercano dentro de cada subset. Más adelante, el algoritmo buscará unos candidatos para determinar si coordenadas de diferentes subsets contienen una distancia más pequeña que la encontrada por el algoritmo de fuerza bruta. \\

Por otro lado, dentro del algoritmo se encuentran variables que nos permiten conocer el número de comparaciones y el tiempo promedio por ejecución ya que estos son los verdaderos resultados que me permiten a mí graficar y observar la complejidad del algorítmo.Cada proceso con la misma cantidad de coordenadas se repite 200 veces, sin embargo, la cantidad de coordenadas va aumentando en un factor de 3/2 comenzando en 500 hasta 1000000. Todos los resultados son plasmados en TXT que contiene 3 diferentes columnas (Cantidad de coordenadas, número de comparaciones promedio y tiempo de ejecución promedio). Con lo cual más adelante se grafico los resultados con la ayuda de Python y me permitió sacar mis conclusiones.