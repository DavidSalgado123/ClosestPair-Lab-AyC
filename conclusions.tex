\section{Conclusión}
Para concluir este laboratorio podemos decir con certeza que logramos el objetivo principal ya que encontramos la complejidad del algoritmo a trabajar. Esto fue analizado en base a la cantidad de coordenadas, la cantidad de comparaciones por ejecución y tiempo de ejecución. Estos valores a simple vista nos presentaron una complejidad lineal como se observa en la gráfica presentada ya que a medidad que aumentan la cantidad de coordenadas por set, es lógico pensar que las comparaciones y el tiempo de ejecución aumentan demostrando que las variables son directamente proporcionales entre ellas. Por otro lado, tambien logramos encontrar la complejidad del algoritmo de manera matemática con lo aprendido en clase y el uso de la Master Equation, estos resultados al igual que el esperado obtuvimos el mismo resultado de que la complejidad del algoritmo realizado para el problema planteado tiene una complejidad de O(n). \\

Para finalizar, puedo decir de que no se presentaron graves problemas en la realización de este laboratorio, lo más dificil fue realmente plantear la parte lógica en un codigo funcional de JAVA para que pueda proporcionar unos resultados. No obstante, gracias al trabajo realizado en clase otras funciones y subprocedimiento fueron reutilizadas de trabajos anteriores tales como la creación y modificación de TXT, graficar unos datos basado en TXT, Crear un set de numeros aleatorios, entre otros. Para futuras investigaciones puede ser interesante saber que sucede si el set no se divide unicamente en 2 sino que se dividiera en 4 por ejemplo donde las coordenadas en Y ya tendrían su importancia por lo cual sería bueno preguntarse si ¿Entre más divisiones se creen para un mismo set, la complejidad del algoritmo puede aumentar ?.