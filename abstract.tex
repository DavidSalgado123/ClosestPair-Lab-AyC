\begin{abstract}
Desde un principio se nos ha comentado la importancia de la complejidad del tiempo de diferentes algoritmos al estar relacionado con los conditional checks lo cual vendría siendo el objetivo de este laboratorio para un algoritmo creado por nosotros para encontrar el par de puntos más cercanos en almenos 3 o más puntos dentro de un plano cartesiano. Este algoritmo plantea cierta cantidad de puntos y les crea unas coordenadas en x aleatorias las cuales posteriormente serán organizadas de menor a mayor. Al ya estar organizados se utiliza la estratégia Divide and Conquer con lo cual nos permite crear grupos más pequeños y ahí utilizar el algoritmo de Brute Force otorgado por el profesor para así poder encontrar el par más cercano. En este caso solo se creará una división entre las coordenadas para así poder crear 2 grupos. Este proceso se realiza 200 veces con cada set de coordenadas para así sacar el tiempo promedio y posteriormente poder sacar una conclusión que se mostrará más adelante en el informe.
\end{abstract}
